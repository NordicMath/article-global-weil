%%%%%%%%%%%%%%%%%%%%%%%%%%%%%%%%%%%%%%%%%
% Short Sectioned Assignment
% LaTeX Template
% Version 1.0 (5/5/12)
%
% This template has been downloaded from:
% http://www.LaTeXTemplates.com
%
% Original author:
% Frits Wenneker (http://www.howtotex.com)
%
% License:
% CC BY-NC-SA 3.0 (http://creativecommons.org/licenses/by-nc-sa/3.0/)
%
%%%%%%%%%%%%%%%%%%%%%%%%%%%%%%%%%%%%%%%%%

%----------------------------------------------------------------------------------------
%	PACKAGES AND OTHER DOCUMENT CONFIGURATIONS
%----------------------------------------------------------------------------------------

\documentclass[paper=a4, fontsize=11pt]{scrartcl} % A4 paper and 11pt font size

\usepackage[T1]{fontenc} % Use 8-bit encoding that has 256 glyphs
\usepackage{fourier} % Use the Adobe Utopia font for the document - comment this line to return to the LaTeX default
\usepackage[english]{babel} % English language/hyphenation
\usepackage{amsmath,amsfonts,amsthm} % Math packages

\usepackage{graphicx}
\usepackage[colorinlistoftodos]{todonotes}
\usepackage[colorlinks=true, allcolors=blue]{hyperref}


\usepackage{lipsum} % Used for inserting dummy 'Lorem ipsum' text into the template

\usepackage{sectsty} % Allows customizing section commands
\allsectionsfont{\centering \normalfont\scshape} % Make all sections centered, the default font and small caps

\usepackage{fancyhdr} % Custom headers and footers
\pagestyle{fancyplain} % Makes all pages in the document conform to the custom headers and footers
\fancyhead{} % No page header - if you want one, create it in the same way as the footers below
\fancyfoot[L]{} % Empty left footer
\fancyfoot[C]{} % Empty center footer
\fancyfoot[R]{\thepage} % Page numbering for right footer
\renewcommand{\headrulewidth}{0pt} % Remove header underlines
\renewcommand{\footrulewidth}{0pt} % Remove footer underlines
\setlength{\headheight}{13.6pt} % Customize the height of the header

\numberwithin{equation}{section} % Number equations within sections (i.e. 1.1, 1.2, 2.1, 2.2 instead of 1, 2, 3, 4)
\numberwithin{figure}{section} % Number figures within sections (i.e. 1.1, 1.2, 2.1, 2.2 instead of 1, 2, 3, 4)
\numberwithin{table}{section} % Number tables within sections (i.e. 1.1, 1.2, 2.1, 2.2 instead of 1, 2, 3, 4)

%\setlength\parindent{0pt} % Removes all indentation from paragraphs - comment this line for an assignment with lots of text

%----------------------------------------------------------------------------------------
%	TITLE SECTION
%----------------------------------------------------------------------------------------

%\newcommand{\defhl}[1]{\emph{#1}}

\newcommand{\horrule}[1]{\rule{\linewidth}{#1}} % Create horizontal rule command with 1 argument of height

\title{
\normalfont \normalsize
\textsc{Work in progress; do not distribute!} \\ [25pt] % Your university, school and/or department name(s)
\horrule{0.5pt} \\[0.4cm] % Thin top horizontal rule
\LARGE New analogies between local and global zeta functions \\ % The assignment title
\horrule{2pt} \\[0.5cm] % Thick bottom horizontal rule
}

\author{Magnus and Olav Hellebust Haaland; Andreas Holmstrom; Torstein Vik} % Your name

\date{\normalsize\today} % Today's date or a custom date


\begin{document}

\maketitle % Print the title

%----------------------------------------------------------------------------------------
%	PROBLEM 1
%----------------------------------------------------------------------------------------

\begin{abstract}

The primary purpose of this document is to record some numerical patterns observed while experimenting with motivic L-functions. A secondary purpose is to point out some analogies between these patterns (which concern "global" zeta functions) with the Weil conjectures (which are theorems on local zeta functions). As far as we can tell, these analogies are new. In particular, they are distinct from (and possibly unrelated to) the classical stories which compare the global Riemann hypothesis and the global functional equation to the Weil conjectures.

These very informal notes grew out of two separate projects, namely (1) a student project on L-functions and reciprocity laws by the first two authors, and (2) a project on analogies and automated reasoning by the other two.


\end{abstract}


\newpage

Here will be pictures:

(1) Genus 2 curve over F5 vs Dirichlet character mod 5. The points are paired with product 5, because we are working with a variety over F5. In the other picture, the points are paired with product 1, as if we were working over the field F1. Note also the functional equation in both cases, which suggests that the Euler characteristic of any non-trivial Dirichlet character equals 2.

(2) Local and global point count for an elliptic curve with CM.

(3) The weird pair conjecture with 289 and 389 as the two examples (one m and one M).


\newpage

\tableofcontents


\newpage

\section{Philosophical comments}








GAAAAHHHRG. Rewrite this.

It is not at all clear that the language of "the field with one element" is appropriate for the purpose of interpreting our numerical observations.

Things we want to get away from:
\begin{itemize}
\item In some of the classical heuristic stories, the object $\mathrm{Spec} \ \mathbb{Z}$ is not of finite type over $\mathrm{Spec} \ \mathbb{F}_1$. Can we get away completely from the need for a base change functor?
\item A global L-function has infinitely many zeroes, and hence any attempt to find a Poincaré duality explaining the functional equation will have to be expressed using infinite-dimensional cohomology groups. Can one get around this issue?

\item There is some overlap between the set of


There are three operations on Dirichlet series, and it is not at all clear how these operations relate to operations on the dark schemes or on the Grothendieck gadgets.

Addition: Mention Dirichlet L-function decomposition into Hurwitz, class group decomposition of Dedekind zeta, and eta quotient sum decompositions.



Introduce the idea of a Grothendieck gadget. It is an object which for any positive integer has a degree $d$ point count.

\end{itemize}


\section{Introduction}

Consider the following abstract data.

\begin{itemize}
\item[Z1] An infinite sequence
$$ a_1, a_2, a_3, \ldots  $$
of complex numbers. We write $a$ for the entire sequence and (by slight abuse of notation) we write $a(t)$ for the formal generating series
$$ 1 + a_1 t + a_2 t^2 + a_3 t^3 + \ldots $$
\item[Z2] A Grothendieck gadget $G$. We write
$$ b_1, b_2, b_3, \ldots   $$
for the associated point counts, and we denote this entire sequence by $b$.
\end{itemize}

The pair $(a, G)$ is called a \emph{zeta object} if the following two axioms are satisfied.
\begin{itemize}
\item[A1] The formal series $a(t)$ has some non-zero radius of convergence $r$, and it extends to a meromorphic function, denoted by $a(z)$, on the entire complex plane. We write $\alpha_1, \alpha_2, \ldots$ for the poles of this function (ordered first by absolute value and if necessary by argument), and similarly we write $\beta_1, \beta_2, \ldots$ for the zeroes.
\item[A2] The sequence $a$ is the Euler transform of the sequence $b$.
\end{itemize}

We may also have laws that give constraints on the function $a(z)$ (and hence on the original sequence $a$). Such a law might be valid for a single zeta object or for some family of zeta objects, depending on the context. In particular, we might have:

\begin{itemize}
\item[L1] A law that reconstructs the function $a(z)$ from the poles and the zeroes together with some auxiliary data $D$.
\item[L2] A law which specifies a symmetry possessed by the sets of zeroes and poles.
\item[L3] A law which specifies the absolute values and/or the arguments of the poles and the zeroes of $a(t)$.
\item[L4] A law which connects the poles and the zeroes to the gadget $G$.
\end{itemize}



\section{Notes to include}

For a local zeta function $Z(t)$ associated to a smooth projective variety $X$ over $\mathbb{F}_q$, we have the functional equation
$$  Z \left( \frac{1}{q^d t} \right) = \varepsilon q^{\frac{dE}{2}} t^E Z(t) $$
where $d$ is the dimension of $X$, $E$ is the Euler characteristic, and $\varepsilon$ is some sign (+1 or -1).

Set $q=1$ in the above functional equation. This gives the equation
$$  Z \left( \frac{1}{t} \right) = \varepsilon t^E Z(t) $$
which is in perfect agreement with the $m$-function of a Dirichlet character, which satisfies
$$  m \left( \frac{1}{t} \right) = -\chi(-1) t^2 m(t) $$
The analogy suggests that $-\chi(-1)$ should be viewed as the sign of the functional equation, while the Euler characteristic of any non-trivial Dirichlet character should be 2.

\section{Examples}





%----------------------------------------------------------------------------------------

\end{document}
