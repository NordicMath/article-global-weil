%%%%%%%%%%%%%%%%%%%%%%%%%%%%%%%%%%%%%%%%%
% Short Sectioned Assignment
% LaTeX Template
% Version 1.0 (5/5/12)
%
% This template has been downloaded from:
% http://www.LaTeXTemplates.com
%
% Original author:
% Frits Wenneker (http://www.howtotex.com)
%
% License:
% CC BY-NC-SA 3.0 (http://creativecommons.org/licenses/by-nc-sa/3.0/)
%
%%%%%%%%%%%%%%%%%%%%%%%%%%%%%%%%%%%%%%%%%

%----------------------------------------------------------------------------------------
%	PACKAGES AND OTHER DOCUMENT CONFIGURATIONS
%----------------------------------------------------------------------------------------

\documentclass[paper=a4, fontsize=11pt]{scrartcl} % A4 paper and 11pt font size

\usepackage[T1]{fontenc} % Use 8-bit encoding that has 256 glyphs
\usepackage{fourier} % Use the Adobe Utopia font for the document - comment this line to return to the LaTeX default
\usepackage[english]{babel} % English language/hyphenation
\usepackage{amsmath,amsfonts,amsthm} % Math packages

\usepackage{graphicx}
\usepackage[colorinlistoftodos]{todonotes}
\usepackage[colorlinks=true, allcolors=blue]{hyperref}


\usepackage{lipsum} % Used for inserting dummy 'Lorem ipsum' text into the template

\usepackage{sectsty} % Allows customizing section commands
\allsectionsfont{\centering \normalfont\scshape} % Make all sections centered, the default font and small caps

\usepackage{fancyhdr} % Custom headers and footers
\pagestyle{fancyplain} % Makes all pages in the document conform to the custom headers and footers
\fancyhead{} % No page header - if you want one, create it in the same way as the footers below
\fancyfoot[L]{} % Empty left footer
\fancyfoot[C]{} % Empty center footer
\fancyfoot[R]{\thepage} % Page numbering for right footer
\renewcommand{\headrulewidth}{0pt} % Remove header underlines
\renewcommand{\footrulewidth}{0pt} % Remove footer underlines
\setlength{\headheight}{13.6pt} % Customize the height of the header

\numberwithin{equation}{section} % Number equations within sections (i.e. 1.1, 1.2, 2.1, 2.2 instead of 1, 2, 3, 4)
\numberwithin{figure}{section} % Number figures within sections (i.e. 1.1, 1.2, 2.1, 2.2 instead of 1, 2, 3, 4)
\numberwithin{table}{section} % Number tables within sections (i.e. 1.1, 1.2, 2.1, 2.2 instead of 1, 2, 3, 4)

%\setlength\parindent{0pt} % Removes all indentation from paragraphs - comment this line for an assignment with lots of text

%----------------------------------------------------------------------------------------
%	TITLE SECTION
%----------------------------------------------------------------------------------------

%\newcommand{\defhl}[1]{\emph{#1}}

\newcommand{\horrule}[1]{\rule{\linewidth}{#1}} % Create horizontal rule command with 1 argument of height

\title{
\normalfont \normalsize
%\textsc{Work in progress; do not distribute!} \\ [25pt] % Your university, school and/or department name(s)
\horrule{0.5pt} \\[0.4cm] % Thin top horizontal rule
\LARGE New analogies between local and global zeta functions \\ % The assignment title
\horrule{2pt} \\[0.5cm] % Thick bottom horizontal rule
}

\author{Magnus and Olav Hellebust Haaland; Andreas Holmstrom; Torstein Vik} % Your name

\date{\normalsize\today} % Today's date or a custom date


\begin{document}

\maketitle % Print the title

%----------------------------------------------------------------------------------------
%	PROBLEM 1
%----------------------------------------------------------------------------------------

\begin{abstract}

REWRITE: The primary purpose of this note is to record some numerical patterns observed while experimenting with motivic L-functions. A secondary purpose is to point out some analogies between these patterns (which concern \emph{global} zeta functions) with the Weil conjectures (which are theorems on \emph{local} zeta functions). As far as we can tell, these analogies are new. In particular, they are distinct from (and possibly unrelated to) the classical stories which compare the global Riemann hypothesis and the global functional equation to the Weil conjectures.

These very informal notes grew out of two separate projects, namely (1) a student project on L-functions and reciprocity laws by the first two authors, and (2) a project on analogies and automated reasoning by the other two.


\end{abstract}

\tableofcontents

\newpage

Here will be pictures:

(1) Genus 2 curve over F5 vs Dirichlet character mod 5. The points are paired with product 5, because we are working with a variety over F5. In the other picture, the points are paired with product 1, as if we were working over the field F1. Note also the functional equation in both cases, which suggests that the Euler characteristic of any non-trivial Dirichlet character equals 2.

(2) Local and global point count for an elliptic curve with CM.

(3) The weird pair conjecture with 289 and 389 as the two examples (one m and one M).



\newpage

\section{Introduction}

The analogy between number fields and function fields, and the closely related analogy between global zeta functions and local zeta functions, are among the main sources of mysteries in arithmetic geometry.

Local zeta functions on one hand are well understood by the theorems known as the Weil conjectures, together with the geometric framework given by Grothendieck's theory of schemes and motives over finite fields.

Global zeta functions (and their constituent pieces referred to as \emph{L-functions}) on the other hand are poorly understood. Many important conjectures are unproven, and we lack a geometric framework playing a role analogous to Grothendieck's theory of schemes.

The hoped-for but still missing geometric framework for the global setting is sometimes referred to as \emph{geometry over the field with one element}, or \emph{$\mathbb{F}_1$-geometry}. Many suggestions have been put forward for how to build such a theory (give refs), but it seems like not a single one of these attempts has led to even a single lemma saying something new about a single global zeta function. In other words, although these theories may be quite sophisticated and useful for certain purposes, they are (\emph{at least in their current state}) inadequate for the specific purpose of better understanding global zeta functions.

Another (equally fundamental) problem with existing approaches is that insofar as they relate to global zeta functions at all, they only seem to relate to the zeta functions for which the constituent L-functions are of $GL_1$ type. As an example of this phenomenon (which is related to the meaning of \emph{finite type} in $\mathbb{F}_1$-geometry), see Theorem 0.1 of Borger (ref, p. 3) and the subsequent remarks made there, saying that only abelian Artin-Tate motives can be defined over $\mathbb{F}_1$. %(In the philosophy of the Langlands program, every L-function is of $GL_n$ type for some positive integer $n$, and the examples of $GL_1$ type are the simplest ones).

In the present note, we take as our starting point the naive idea that (1) the sequence of coefficients of a local zeta function and (2) the sequence of coefficients of a global zeta function (or L-function) \emph{should be considered on an equal footing}. From the viewpoint of traditional scheme theory, this doesn't make any sense, but the point is that we might be working towards a new geometric theory which in the local case specializes to the theory of schemes, but which in the global case doesn't even admit a base-change functor to schemes over $\mathrm{Spec} \ \mathbb{Z}$.

Based on numerical patterns in a number of examples, we postulate the existence of new objects called \emph{dark schemes}, and make some observations suggesting that a good theory of dark schemes may circumvent some of the finiteness problems seemingly inherent in other approaches to the field with one element, while allowing for some pleasant unifying analogies between local zeta functions and global zeta functions, crucially including some examples of $GL_2$ type. All of this is conveniently expressed in terms of a simple book-keeping tool we call Euler-Leibniz-Newton (or ELN) triples.

We begin with the introduction of ELN triples and a very brief review of zeta functions, before giving some key examples of the patterns appearing for local zeta functions, for global zeta functions of $GL_1$ type and for global zeta functions of $GL_2$ type. In the last section we collect some philosophical remarks and directions for future study.





\section{Zeta functions and L-functions}

We assume that the reader is familiar with (or is willing to read about) the following constructions.

\begin{enumerate}

\item When $X$ is a scheme which is flat and of finite type over $\mathrm{Spec} \ \mathbb{Z}$, then the Hasse-Weil zeta function associated to $X$ is defined by
$$ Z_X(s) = \prod_{x \in X_{cl}} \left( 1 - N(x)^{-s}   \right)^{-1}  $$
Here $X_{cl}$ is the set of closed points of the scheme $X$, and for any closed point $x$, we have written $N(x)$ for the number of elements in the residue field of $x$. See \S 6.5 of \cite{Mustata} for more details.
\item For any motive $M$ over $\mathbb{Q}$, we have the motivic L-function $L(M, s)$. Reference: \S 4.5 of \cite{Farmer17}.
\item When $V$ is a variety over a finite field $\mathbb{F}_q$, we have a zeta function

\end{enumerate}

For a concrete example, consider the projective scheme $X$ defined by the equation
$  y^2 = \ldots $


Hasse-Weil zeta functions in the local and global case. Motivic L-functions. Definition of $m$-functions.


\section{The idea of $m$-functions}




\section{ELN triples and the idea of dark schemes}

Define Euler transform, Newton transform, Leibniz transform, and ELN triple. Give examples.


\section{Then local (and hence rational) case}


\section{The global but still rational case (a.k.a. $GL_1$}


\section{The global non-rational case (examples from $GL_2$)}



\section{Remarks and questions}



%----------------------------------------------------------------------------------------

\end{document}
